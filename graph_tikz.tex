\documentclass{article}
\usepackage[T2A]{fontenc} % Поддержка Кириллицы
\usepackage[utf8]{inputenc} % Поддержка utf8
\usepackage{tikz}
\usetikzlibrary{snakes,arrows,shapes}
\usepackage{amsmath}
\usepackage[russian]{babel} % Поддержка русского языка
\usepackage{paralist}
\usetikzlibrary{fit}
\usepackage[export]{adjustbox}

\begin{document}
\pagestyle{empty}


%Начало задания
\noindent\textbf{Задание.} Дан ориентированный граф $G = (V, E)$, где
$V$ = \{1, 2, 3, 4, 5 \} и $E$ = \{(1, 5), (2, 1), (5, 2), (5, 5) \}.\\
Сколько всего ребер в метаграфе орграфа $G$?
\vspace{2mm}

\noindent\begin{inparaenum}[(1)]
\item 6 или более;
 \item 5; \item 4; \item 3; \item 2; \item 1; \item 0;\end{inparaenum}\\



\noindent\textbf{Ответ:} (7).\\

%Конец задания


\noindent\textbf{Подробное обоснование.}
Исходный орграф $G$:
\vspace{8mm}

% Start of code
\begin{adjustbox}{max size={1.4\textwidth}{0,7\textheight}, center} % ограничиваем ширину и высоту
\begin{tikzpicture}[>=latex',line join=bevel,]
  \pgfsetlinewidth{1bp}
%
\pgfsetcolor{black}
% Edge: 2 -> 1
\draw [->] (128.530bp,159.700bp) .. controls (127.470bp,152.320bp) and (126.220bp,143.520bp) .. (123.440bp,124.100bp);% Edge: 1 -> 5
\draw [->] (121.250bp,87.697bp) .. controls (121.350bp,80.407bp) and (121.480bp,71.726bp) .. (121.760bp,52.104bp);% Edge: 5 -> 2
\draw [->] (135.370bp,49.996bp) .. controls (143.360bp,59.991bp) and (152.750bp,73.849bp) .. (157.000bp,88.000bp) .. controls (161.600bp,103.320bp) and (160.620bp,108.410bp) .. (157.000bp,124.000bp) .. controls (154.850bp,133.270bp) and (150.840bp,142.780bp) .. (141.240bp,161.020bp);% Edge: 5 -> 5
\draw [->] (141.900bp,46.432bp) .. controls (154.690bp,49.675bp) and (167.000bp,45.531bp) .. (167.000bp,34.000bp) .. controls (167.000bp,25.982bp) and (161.050bp,21.536bp) .. (141.900bp,21.568bp);
\node[ellipse, draw, minimum width=54bp, minimum height=36bp] (v3) at (43.00 bp,178.00 bp) {3};\node (v3) at (43.00 bp,178.00 bp) [ellipse] {};
\node[ellipse, draw, minimum width=54bp, minimum height=36bp] (v2) at (131.00 bp,178.00 bp) {2};\node (v2) at (131.00 bp,178.00 bp) [ellipse] {};
\node[ellipse, draw, minimum width=54bp, minimum height=36bp] (v1) at (121.00 bp,106.00 bp) {1};\node (v1) at (121.00 bp,106.00 bp) [ellipse] {};
\node[ellipse, draw, minimum width=54bp, minimum height=36bp] (v5) at (122.00 bp,34.00 bp) {5};\node (v5) at (122.00 bp,34.00 bp) [ellipse] {};
\node[ellipse, draw, minimum width=54bp, minimum height=36bp] (v4) at (219.00 bp,178.00 bp) {4};\node (v4) at (219.00 bp,178.00 bp) [ellipse] {};
 \node[draw, thick, blue, dashed, rounded corners=25pt, fit=(v3) , inner sep=25pt] {};
 \node[draw, thick, blue, dashed, rounded corners=25pt, fit=(v2) (v1) (v5) , inner sep=25pt] {};
 \node[draw, thick, blue, dashed, rounded corners=25pt, fit=(v4) , inner sep=25pt] {};
\end{tikzpicture}
\end{adjustbox}
% End of code
\\\\Обращение $G_r$ орграфа $G$:
\vspace{2mm}

% Start of code
\begin{adjustbox}{max size={1.4\textwidth}{0,7\textheight}, center} % ограничиваем ширину и высоту
\begin{tikzpicture}[>=latex',line join=bevel,]
  \pgfsetlinewidth{1bp}
%
\pgfsetcolor{black}
% Edge: 2 -> 5
\draw [->] (135.400bp,160.050bp) .. controls (133.350bp,152.490bp) and (130.870bp,143.370bp) .. (125.560bp,123.790bp);% Edge: 1 -> 2
\draw [->] (153.370bp,49.996bp) .. controls (161.360bp,59.991bp) and (170.750bp,73.849bp) .. (175.000bp,88.000bp) .. controls (179.600bp,103.320bp) and (179.600bp,108.680bp) .. (175.000bp,124.000bp) .. controls (171.910bp,134.280bp) and (166.110bp,144.410bp) .. (153.370bp,162.000bp);% Edge: 5 -> 1
\draw [->] (125.600bp,88.055bp) .. controls (127.650bp,80.489bp) and (130.130bp,71.372bp) .. (135.440bp,51.789bp);% Edge: 5 -> 5
\draw [->] (140.900bp,118.430bp) .. controls (153.690bp,121.680bp) and (166.000bp,117.530bp) .. (166.000bp,106.000bp) .. controls (166.000bp,97.982bp) and (160.050bp,93.536bp) .. (140.900bp,93.568bp);
\node[ellipse, draw, minimum width=54bp, minimum height=36bp] (v3) at (43.00 bp,178.00 bp) {3};\node (v3) at (43.00 bp,178.00 bp) [ellipse] {};
\node[ellipse, draw, minimum width=54bp, minimum height=36bp] (v2) at (140.00 bp,178.00 bp) {2};\node (v2) at (140.00 bp,178.00 bp) [ellipse] {};
\node[ellipse, draw, minimum width=54bp, minimum height=36bp] (v5) at (121.00 bp,106.00 bp) {5};\node (v5) at (121.00 bp,106.00 bp) [ellipse] {};
\node[ellipse, draw, minimum width=54bp, minimum height=36bp] (v1) at (140.00 bp,34.00 bp) {1};\node (v1) at (140.00 bp,34.00 bp) [ellipse] {};
\node[ellipse, draw, minimum width=54bp, minimum height=36bp] (v4) at (237.00 bp,178.00 bp) {4};\node (v4) at (237.00 bp,178.00 bp) [ellipse] {};
 \node[draw, thick, blue, dashed, rounded corners=25pt, fit=(v3) , inner sep=25pt] {};
 \node[draw, thick, blue, dashed, rounded corners=25pt, fit=(v2) (v1) (v5) , inner sep=25pt] {};
 \node[draw, thick, blue, dashed, rounded corners=25pt, fit=(v4) , inner sep=25pt] {};
\end{tikzpicture}
\end{adjustbox}
% End of code
 \vspace{4mm} 

 Обойдем $G_r$ в глубину и получим список его вершин в порядке убывания их post-значений: \{4, 3, 1, 2, 5 \}.\\Идя по списку, из каждой (ранее не посещенной) вершины обойдем $исходный$ орграф в глубину.
Вершины, посещаемые при каждом новом обходе, будут давать отдельную ССК.
\\(Здесь и далее ССК – сильно связная компонента).\\
\begin{itemize}
\item $visit($4$)$ дает ССК: $\langle 4 \rangle$\item $visit($3$)$ дает ССК: $\langle 3 \rangle$\item $visit($1$)$ дает ССК: $\langle 1, 5, 2 \rangle$
\end{itemize}

Сформируем из каждой ССК метавершину и будем соединять направленным ребром пару метавершин $X$ и $Y$,
если в метавершине $X$ есть вершина, из которой идет ребро в вершину, лежащую в метавершине $Y$.
Таким образом получим метаграф $G_r$ орграфа $G$:\\\\

% Start of code
\begin{adjustbox}{max size={1.4\textwidth}{0,4\textheight}, center} % ограничиваем ширину и высоту
\begin{tikzpicture}[>=latex',line join=bevel,]
  \pgfsetlinewidth{1bp}
%
\pgfsetcolor{black}

\node[ellipse, draw, minimum width=54bp, minimum height=36bp] (v) at (27.00 bp,18.00 bp) {$\langle3\rangle$};
\node[ellipse, draw, minimum width=54bp, minimum height=36bp] (v) at (102.00 bp,18.00 bp) {$\langle1,2,5\rangle$};
\node[ellipse, draw, minimum width=54bp, minimum height=36bp] (v) at (177.00 bp,18.00 bp) {$\langle4\rangle$};
\end{tikzpicture}
\end{adjustbox}
% End of code
\vspace{1em}

В этом метаграфе ровно \textbf{0 рёбер}.

\end{document}

